% !TeX root = RJwrapper.tex
\title{ino: Initialization of Numerical Optimization in R}
\author{by Lennart Oelschläger and Marius Ötting}

\maketitle

\abstract{%
Numerical optimization of some target functions is of great relevance in many fields, as many optimization problems cannot be solved analytically. While R provides several functions for applying numerical optimization, such as \texttt{nlm()} or \texttt{optim()} from the stats package, users have to select initial values when applying such methods. Since the choice of initial values can hugely affect the convergence time and rate, several strategies exist for selecting initial values. The ino package provides a toolbox for evaluating the effect of the initial values on the optimization as well as comparing different initialization strategies and optimizers.
}

\hypertarget{introduction}{%
\section{Introduction}\label{introduction}}

\hypertarget{a-brief-background-on-numerical-optimization}{%
\section{A brief background on numerical optimization}\label{a-brief-background-on-numerical-optimization}}

\includegraphics{ino_files/figure-latex/plot_ackley-1.pdf}

\hypertarget{numerical-optimization}{%
\subsection{Numerical optimization}\label{numerical-optimization}}

\hypertarget{strategies-for-choosing-initial-values}{%
\subsection{Strategies for choosing initial values}\label{strategies-for-choosing-initial-values}}

\hypertarget{usage-of-ino}{%
\section{Usage of \{ino\}}\label{usage-of-ino}}

\hypertarget{implementation-of-strategies-for-choosing-initial-values}{%
\subsection{Implementation of strategies for choosing initial values}\label{implementation-of-strategies-for-choosing-initial-values}}

\hypertarget{application-to-mixture-models}{%
\subsection{Application to mixture models}\label{application-to-mixture-models}}

\hypertarget{application-to-hidden-markov-models}{%
\subsection{Application to hidden Markov models}\label{application-to-hidden-markov-models}}

\hypertarget{application-to-multinomial-probit-models}{%
\subsection{Application to multinomial probit models}\label{application-to-multinomial-probit-models}}

\hypertarget{discussion}{%
\section{Discussion}\label{discussion}}

\bibliography{RJreferences.bib}

\address{%
Lennart Oelschläger\\
Bielefeld University\\%
Department of Business Administration and Economics\\ Bielefeld, Germany\\
%
\url{https://loelschlaeger.de/}\\%
%
\href{mailto:lennart.oelschlaeger@uni-bielefeld.de}{\nolinkurl{lennart.oelschlaeger@uni-bielefeld.de}}%
}

\address{%
Marius Ötting\\
Bielefeld University\\%
Department of Business Administration and Economics\\ Bielefeld, Germany\\
%
\url{https://www.uni-bielefeld.de/fakultaeten/wirtschaftswissenschaften/lehrbereiche/stats/team/marius-otting-(m.sc}.)/\\%
\textit{ORCiD: \href{https://orcid.org/0000-0002-9373-0365}{0000-0002-9373-0365}}\\%
\href{mailto:marius.oetting@uni-bielefeld.de}{\nolinkurl{marius.oetting@uni-bielefeld.de}}%
}
